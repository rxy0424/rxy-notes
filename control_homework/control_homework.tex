\documentclass[UTF8]{ctexart}

\usepackage{theorem}
\usepackage{amsmath}
\usepackage{amssymb}
\usepackage{mathtools}
\usepackage[]{mcode}


\title{任昕旸的最优控制作业}
\author{任昕旸\\ 16B904014}

\begin{document}
\maketitle

\section{3.7}
本题存在错误,改正后的题目为:求泛函
\[J({{x}_{1}},{{x}_{2}})=\int_{0}^{5}{\sqrt{1+{{{\dot{x}}}_{1}}^{2}\left( t \right)+{{{\dot{x}}}_{2}}^{2}\left( t \right)}}dt\]
的极值,已知${{x}_{1}}\left( 0 \right)=-5$,${{x}_{1}}\left( 5 \right)=0$,${{x}_{2}}\left( 0 \right)=0$,${{x}_{2}}\left( 5 \right)=\pi $,${{x}_{1}}^{2}\left( t \right)+{{t}^{2}}=25$。
	解:由题目条件
		\[\left\{ \begin{array}{*{35}{l}}
   {{x}_{1}}^{2}\left( t \right)+{{t}^{2}}=25  \\
   {{x}_{1}}\left( 0 \right)=-5  \\
   {{x}_{1}}\left( 5 \right)=0  \\
\end{array} \right.\]
		可得:
		\[{{x}_{1}}\left( t \right)=-\sqrt{25-{{t}^{2}}}\]
\[\therefore {{\dot{x}}_{1}}(t)=\frac{t}{\sqrt{25-{{t}^{2}}}}\]
		将其带入原泛函可得:
\[J({{x}_{1}})=\int_{0}^{5}{\sqrt{1+\frac{{{t}^{2}}}{25-{{t}^{2}}}+{{{\dot{x}}}_{2}}^{2}\left( t \right)}}dt\] 其中\[{{x}_{2}}(0)=0,{{x}_{2}}(5)=\pi \]
所以简化为求\[{{x}_{2}}\]的极值轨线。
由欧拉方程\[{{L}_{x}}-\frac{d}{dt}{{L}_{{\dot{x}}}}=0\]可得:
\[{{L}_{{{x}_{2}}}}-\frac{d}{dt}{{L}_{{{{\dot{x}}}_{2}}}}=-\frac{d}{dt}\left( \frac{2{{{\dot{x}}}_{2}}\left( t \right)}{\sqrt{1+\frac{{{t}^{2}}}{25-{{t}^{2}}}+{{{\dot{x}}}_{2}}^{2}\left( t \right)}} \right)=0\]
		 \[\therefore \frac{2{{{\dot{x}}}_{2}}\left( t \right)}{\sqrt{1+\frac{{{t}^{2}}}{25-{{t}^{2}}}+{{{\dot{x}}}_{2}}^{2}\left( t \right)}}=c\]
		 \[{{\dot{x}}^{2}}_{2}\left( t \right)=\frac{c'}{25-{{t}^{2}}}\]
${{x}_{2}}(t)={{c}_{1}}\arcsin (t/5)+{{c}_{2}}$
代入边界条件解得
\[\left\{ \begin{matrix}
   {{c}_{1}}=2  \\
   {{c}_{2}}=0  \\
\end{matrix} \right.\]
		可以得到极值轨线:
${x}^{*}_{2}(t)=2\arcsin (t/5)$
${{\dot{x}}_{2}}(t)=\frac{2}{\sqrt{25-{{t}^{2}}}}$
对应的泛函极值为:
\[J({{x}_{1}})=\int_{0}^{5}{\sqrt{1+\frac{{{t}^{2}}}{25-{{t}^{2}}}+\frac{4}{25-{{t}^{2}}}}}dt=\frac{\sqrt{29}}{2}\pi \]
\section{3.10}
确定泛函
\[J(x)={\int_{0}^{4}{{{\left[ \dot{x}\left( t \right)-1 \right]}^{2}}\left[ \dot{x}\left( t \right)+1 \right]}^{2}}dt\]
的极值轨线和极值,假设极值轨线最多只存在一个角点,已知$x\left( 0 \right)=0$,$x\left( 4 \right)=6$。

解:由欧拉方程可得:
		\[{{L}_{x}}-\frac{d}{dt}{{L}_{{\dot{x}}}}=0-4\left( 3{{{\dot{x}}}^{2}}\left( t \right)\ddot{x}\left( t \right)-\ddot{x}\left( t \right) \right)=0\]
		由此可得:
${{x}_{1}}\left( t \right)={{c}_{1}}t+{{c}_{2}}$
${{x}_{2}}\left( t \right)=\frac{\sqrt{3}}{3}t+{{c}_{5}}$
${{x}_{3}}\left( t \right)=-\frac{\sqrt{3}}{3}t+{{c}_{6}}$

(1)	当极值轨线不存在角点时,
由边界条件可知, ${{x}_{2}}\left( t \right)$与${{x}_{3}}\left( t \right)$不符合,所以选择${{x}_{1}}\left( t \right)$。
由边界条件可得
\[\left\{ \begin{array}{*{35}{l}}
   x\left( 0 \right)={{c}_{2}}=0  \\
   x\left( 4 \right)=4{{c}_{1}}+{{c}_{2}}=6  \\
\end{array} \right.\]
\[\therefore \left\{ \begin{array}{*{35}{l}}
   {{c}_{1}}=\frac{3}{2}  \\
   {{c}_{2}}=0  \\
\end{array} \right.\]
$x\left( t \right)=\frac{3}{2}t$
$J(x)={{\int_{0}^{4}{{{\left[ \frac{3}{2}-1 \right]}^{2}}\left[ \frac{3}{2}+1 \right]}}^{2}}dt=\frac{25}{4}$

(2)		当极值轨线存在一个角点时,
$x\left( t \right)=\left\{ \begin{matrix}
   {{c}_{1}}t+{{c}_{2}}\left( 0\le t\le {{t}_{1}} \right)  \\
   {{c}_{3}}t+{{c}_{4}}\left( {{t}_{1}}\le t\le 4 \right)  \\
\end{matrix} \right.$
\[\dot{x}\left( {{t}_{1}}^{-} \right)={{c}_{1}}\]\[\dot{x}\left( {{t}_{1}}^{+} \right)={{c}_{3}}\]
由角点条件
\[\left\{ \begin{array}{*{35}{l}}
   {{\left. {{L}_{{\dot{x}}}} \right|}_{{{t}_{1}}^{-}}}={{\left. {{L}_{{\dot{x}}}} \right|}_{{{t}_{1}}^{+}}}  \\
   {{\left. \left( L-{{L}_{{\dot{x}}}}\dot{x} \right) \right|}_{{{t}_{1}}^{-}}}={{\left. \left( L-{{L}_{{\dot{x}}}}\dot{x} \right) \right|}_{{{t}_{1}}^{+}}}  \\
\end{array} \right.\]
可得:
\[\left\{ \begin{array}{*{35}{l}}
   4\left( {{c}_{1}}^{3}-{{c}_{1}} \right)=4\left( {{c}_{3}}^{3}-{{c}_{3}} \right)  \\
   {{\left( {{c}_{1}}^{2}-1 \right)}^{2}}-4\left( {{c}_{1}}^{3}-{{c}_{1}} \right){{c}_{1}}={{\left( {{c}_{3}}^{2}-1 \right)}^{2}}-4\left( {{c}_{3}}^{3}-{{c}_{3}} \right){{c}_{3}}  \\
\end{array} \right.\]
由边界条件可得:
\[\left\{ \begin{array}{*{35}{l}}
   {{c}_{2}}=0  \\
   4{{c}_{3}}+{{c}_{4}}=6  \\
\end{array} \right.\]
由状态连续性可得:
${{c}_{1}}{{t}_{1}}+{{c}_{2}}={{c}_{3}}{{t}_{1}}+{{c}_{4}}$
联立上述五个式子,解得:
\[\left\{ \begin{array}{*{35}{l}}
   {{c}_{1}}=1  \\
   {{c}_{2}}=0  \\
   {{c}_{3}}=-1  \\
   {{c}_{4}}=10  \\
   {{t}_{1}}=5  \\
\end{array} \right.\]
或者
\[\left\{ \begin{array}{*{35}{l}}
   {{c}_{1}}=-1  \\
   {{c}_{2}}=0  \\
   {{c}_{3}}=1  \\
   {{c}_{4}}=2  \\
   {{t}_{1}}=-1  \\
\end{array} \right.\]
			可以看出角点不在积分区间,应舍去此种情况,终上:
			极值轨线:
${{x}^{*}}\left( t \right)=\frac{3}{2}t$
			泛函极值:
						$J(x)={{\int_{0}^{4}{{{\left[ \frac{3}{2}-1 \right]}^{2}}\left[ \frac{3}{2}+1 \right]}}^{2}}dt=\frac{25}{4}$
\section{3.18}
求使泛函
$J(x)={{\int_{0}^{{{t}_{f}}}{u}}^{2}}\left( t \right)dt$
达到极值的最优控制${{u}^{*}}(t)$和相应的最优轨线${{x}^{*}}(t)$。已知
\[\dot{x}\left( t \right)=\left[ \begin{matrix}
   0 & 1  \\
   0 & 0  \\
\end{matrix} \right]x\left( t \right)+\left[ \begin{matrix}
   0  \\
   1  \\
\end{matrix} \right]u\left( t \right)\]
边界条件和终端约束为
$x\left( 0 \right)={{\left[ \begin{matrix}
   1 & 1  \\
\end{matrix} \right]}^{T}}$,${{x}_{1}}\left( {{t}_{f}} \right)=-{{t}_{f}}^{2}$,${{x}_{2}}\left( {{t}_{f}} \right)=0$

	解:$H=L+{{\lambda }^{T}}f={{u}^{2}}\left( t \right)+{{\lambda }_{1}}x{}_{2}\left( t \right)+{{\lambda }_{2}}u\left( t \right)$

	(1)由状态方程\[\dot{x}=f=\frac{\partial H}{\partial \lambda }\]可得: 
\[\left\{ \begin{array}{*{35}{l}}
   {{{\dot{x}}}_{1}}={{x}_{2}}  \\
   {{{\dot{x}}}_{2}}=u  \\
\end{array} \right.\]

(2)由共轭方程\[\dot{\lambda }=-\frac{\partial H}{\partial x}\]可得:
\[\left\{ \begin{array}{*{35}{l}}
   {{{\dot{\lambda }}}_{1}}=-{{H}_{{{x}_{1}}}}=0  \\
   {{{\dot{\lambda }}}_{2}}=-{{H}_{{{x}_{2}}}}=-{{\lambda }_{1}}  \\
\end{array} \right.\]

	(3)由控制方程$\frac{\partial H}{\partial u}=0$可得:
			${{H}_{u}}=2u+{{\lambda }_{2}}=0$

	(4)又因为$\phi =0$,${{\psi }_{1}}={{x}_{1}}\left( {{t}_{f}} \right)+{{t}_{f}}^{2}=0$
	    由边界条件${{\left. \lambda \left( {{t}_{f}} \right)=\frac{\partial \phi }{\partial x}+\frac{\partial {{\psi }^{T}}}{\partial x}\gamma  \right|}_{{{t}_{f}}}}$可得:
		${{\left. {{\lambda }_{1}}\left( {{t}_{f}} \right)=0+\frac{\partial }{\partial {{x}_{1}}}\left( {{x}_{1}}\left( {{t}_{f}} \right)+{{t}_{f}}^{2} \right){{\gamma }_{1}} \right|}_{{{t}_{f}}}}={{\gamma }_{1}}\left( {{t}_{f}} \right)$
		且${{x}_{1}}\left( 0 \right)=1$,${{x}_{2}}\left( 0 \right)=1$,${{x}_{2}}\left( {{t}_{f}} \right)=0$

	(5)由哈密顿函数终端要求$H\left( {{t}_{f}} \right)=-\frac{\partial \phi }{\partial {{t}_{f}}}-{{\gamma }^{T}}\left( {{t}_{f}} \right)\frac{\partial \psi }{\partial {{t}_{f}}}$可得:
		\[H\left( {{t}_{f}} \right)=-{{\gamma }_{1}}\left( {{t}_{f}} \right)\left( {{{\dot{x}}}_{1}}\left( {{t}_{f}} \right)+2{{t}_{f}} \right)\]
		即:\[{{u}^{2}}\left( {{t}_{f}} \right)+{{\lambda }_{1}}\left( t{}_{f} \right)x{}_{2}\left( {{t}_{f}} \right)+{{\lambda }_{2}}\left( {{t}_{f}} \right)u\left( {{t}_{f}} \right)=-{{\gamma }_{1}}\left( {{t}_{f}} \right)\left( {{{\dot{x}}}_{1}}\left( {{t}_{f}} \right)+2{{t}_{f}} \right)\]
		由
		\[\left\{ \begin{array}{*{35}{l}}
   {{{\dot{x}}}_{1}}={{x}_{2}}  \\
   {{{\dot{x}}}_{2}}=u  \\
   {{{\dot{\lambda }}}_{1}}=0  \\
   {{{\dot{\lambda }}}_{2}}=-{{\lambda }_{1}}  \\
   2u+{{\lambda }_{2}}=0  \\
\end{array} \right.\]
以及边界条件${{x}_{1}}\left( 0 \right)=1$和${{x}_{2}}\left( 0 \right)=1$得:
\[\left\{ \begin{array}{*{35}{l}}
   \begin{array}{*{35}{l}}
   {{\lambda }_{1}}=a  \\
\end{array}  \\
   {{\lambda }_{2}}=-at+b  \\
   u=\frac{1}{2}at-\frac{1}{2}b  \\
   {{x}_{1}}=\frac{1}{12}a{{t}^{3}}-\frac{1}{4}b{{t}^{2}}+t+1  \\
   {{x}_{2}}=\frac{1}{4}a{{t}^{2}}-\frac{1}{2}bt+1  \\
\end{array} \right.\]
由终端条件可得
\[\left\{ \begin{array}{*{35}{l}}
   {{x}_{1}}\left( {{t}_{f}} \right)+{{t}_{f}}^{2}=\frac{1}{12}a{{t}_{f}}^{3}-\frac{1}{4}b{{t}_{f}}^{2}+{{t}_{f}}+1+{{t}_{f}}^{2}=0  \\
   {{x}_{2}}\left( {{t}_{f}} \right)=\frac{1}{4}a{{t}_{f}}^{2}-\frac{1}{2}b{{t}_{f}}+1=0  \\
   {{\left( \frac{1}{2}a{{t}_{f}}-\frac{1}{2}b \right)}^{2}}+a\left( \frac{1}{4}a{{t}_{f}}^{2}-\frac{1}{2}b{{t}_{f}}+1 \right)+\left( -a{{t}_{f}}+b \right)\left( \frac{1}{2}a{{t}_{f}}-\frac{1}{2}b \right)=-a\left( \frac{1}{4}a{{t}_{f}}^{2}-\frac{1}{2}b{{t}_{f}}+1+2{{t}_{f}} \right)  \\
\end{array} \right.\]解之得:
$\left\{ \begin{matrix}
   a=17.7883  \\
   b=18.9215  \\
   {{t}_{f}}=2.0159  \\
\end{matrix} \right.$	
所以得最优控制:
				${{u}^{*}}\left( t \right)=8.8941{{t}^{2}}-9.4608$
其对应的最优轨线:
\[{{x}^{*}}\left( t \right)=\left[ \begin{matrix}
   1.4824{{t}^{3}}-4.7304{{t}^{2}}+t+1  \\
   4.4471{{t}^{2}}-9.4608t+1  \\
\end{matrix} \right]\]
\section{程序}
3-10:
\begin{lstlisting}
% 3-10:
% 欧拉方程 D2x*(2*D1x-1)=0;
% 假设方程 x(t)=c1*t+c2, 0<t<t1;
%           x(t)=c3*t+c4, t1<t<4;
% 边界方程  c2=0, 4*c3+c4=6;
%           c1*t1+c2=c3*t1+c4,
% 角点条件: 4*c1*(c1^2-1)=4*c3*(c3^2-1);
% 角点条件:(c1^2-1)^2-4*c1^2*(c1-1)=(c3^2-1)^2-4*c3^2*(c3-1);
[c1,c2,c3,c4,t1]=solve(...
    'c2=0','4*c3+c4=6',...
    'c1*t1+c2=c3*t1+c4','4*c1*(c1^2-1)=4*c3*(c3^2-1)',...
    '(c1^2-1)^2-4*c1^2*(c1^2-1)=(c3^2-1)^2-4*c3^2*(c3^2-1)',...
    'c1','c2','c3','c4','t1'
    )
解得:  c1=-1;		c2=0;  	c3=1;  	c4=2;  	t1=-1;
       c1=1;		c2=0;  	c3=-1;  	c4=10;  	t1=5;
   \end{lstlisting}


3-18:
   \begin{lstlisting}
% 3-18:
% D1x1=x1,D1x2=u
% 2u+lamda2=0
% D1lamda1=0,D1lamda2=-lamda1
% x1(0)=1,x2(0)=1,lamda1(tf)=gama,ladma2(tf)=0
% lamda1=a,lamda2=-1*a*t+b
% x2=0.25*a*tf^2-0.5*b*tf+1,x1=(a*tf^3)/12-0.25*b*tf^2+tf+1
% x1(tf)=-1*tf^2,x2(tf)=0,H(tf)=u(tf)^2+lamda1*x2+lamda2*u=-1*gama*(D1x1(tf)+2*tf)
[a,b,tf]=solve(...
'(a*tf^3)/12-0.25*b*tf^2+tf+1=-1*tf^2',...                      
'0.25*a*tf^2-0.5*b*tf+1=0',...                                        
'(0.5*(a*tf-b))^2+a*(0.25*a*tf^2-0.5*b*tf+1)+(b-a*tf)*0.5*(a*tf-b)=-1*a*(0.25*a*tf^2-0.5*b*tf+1+2*tf)'
) 
解得:  a=17.7883;			b=18.9215;			tf=2.0159;
\end{lstlisting}

\end{document}
