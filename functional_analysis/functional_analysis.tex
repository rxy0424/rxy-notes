\documentclass[UTF8]{ctexbook}

\usepackage{theorem}
\usepackage{amsmath}
\usepackage{amssymb}
\usepackage{mathtools}

\newtheorem{definition}{定义}[chapter]
\newtheorem{thm}{定理}[chapter]

\newcommand{\jhshi}[1]{\overline{\overline{#1}}}

\title{任昕旸同志的泛函分析笔记}
\author{任昕旸}
\date{\today}

\begin{document}
\maketitle
\tableofcontents

\chapter{预备知识}
\section{集合的一般知识}
\subsection{集合及其运算}
特殊集合有全集和空集,集合的关系有交、并补三种。

一个比较重要的公式,De Morgan公式:
\begin{equation}
    (\bigcup_{i \in I} A_i)^C = \bigcap_{i \in I} A_i^C
\end{equation}

\begin{equation}
    (\bigcap_{i \in I} A_i)^C = \bigcup_{i \in I} A_i^C
\end{equation}

还有直积
\begin{equation}
    \prod_{k=1}^n A_k = \{ (x_1,x_2,\cdots,x_n)|x_k \in A_k,k=1,2,3,\cdots,n \}
\end{equation}

\begin{definition}
    设$A$,$B$是两个集合,如果存在由$A$到$B$上的一对一映射$T$,则说$A$与$B$成
    一一对应或称$A$与$B$对等,记为$A \sim B$。
\end{definition}

由定义可知其有三个基本性质:
\begin{enumerate}
    \item 自反性:$A \sim A$
    \item 对称性:若$A \sim B$,则$B \sim B$
    \item 传递性:若$A \sim B, B \sim C$,则$A \sim B$
\end{enumerate}

\subsection{可列集}
\begin{definition}
    若$A$与$B$对等,则说$A$与$B$有相同的势(或基数),记$A$的势为$\overline{\overline A}$,
    $B$的势为$\overline {\overline {B}}$,则$\overline{\overline {A}} = \overline{\overline {B}}$
\end{definition}


\begin{definition}
    对$\alpha \in \Gamma$,称$\{ A_{\alpha} \}_{\alpha \in \Gamma}$为定义在$\Gamma$
    上的集族,若$\Gamma = N$,则称其为集列。
\end{definition}

\begin{definition}
    集族的交
    \begin{equation}
    \bigcap_{\alpha \in \Gamma} A_{\alpha} 
    = \{ x \in X \mid \forall \alpha \in \Gamma ,
        x \in A_{\alpha} \}
    \end{equation}
\end{definition}

\begin{definition}
    集族的并
    \begin{equation}
    \bigcup_{\alpha \in \Gamma} A_{\alpha} 
    = \{ x \in X \mid \exists \alpha \in \Gamma ,
    x \in A_{\alpha} \}
    \end{equation}
\end{definition}

\begin{thm}
    无限集必与它的某个真子集对等.
\end{thm}

\begin{definition}
    凡与自然数集$N$对等的集合称为可列集 (或可数集).
\end{definition}

\begin{thm}
    任意无限集必含有可列子集.
\end{thm}

\begin{thm}
    有限集或可列个可列集的并仍是可列集;可列个有限集的并(若为无限集),也是可列集.
\end{thm}

\begin{thm}
    有一组可列集$\{ A_1, A_2, \cdots, A_n \}$,则直积$ \prod_{k=1}^n A_k $也是可列集.
\end{thm}

\begin{thm}
    点集$ ( 0, 1)  = \{ x \in R \mid 0 <  x <  1 \} $是不可列的无限集.
\end{thm}

\begin{thm}
    设$A$为有限集可列集,$B$是任一无限集,则$\jhshi{A \cup B} = \jhshi B$
\end{thm}

\section{实数的集的基本结构}
\subsection{代数结构}
$N$、$Z$不是数域

\subsection{有序结构}
\begin{definition}
    设$x \in A \subseteq R$,如果$\exists M > 0, |x| \leq M$,则$x$有界.
\end{definition}

\begin{definition}
    若$\exists a, b \in R , \forall x, a \leq x \leq b$,则$a$为下界,$b$为上界.
\end{definition}

\begin{definition}
    设$E \subseteq R$,$\exists \beta \in R$,使
    \begin{enumerate}
        \item $\forall x \in R$有$x \leq \beta$
        \item $\forall  \beta' < \beta$,$\exists x_0 > \beta'$
    \end{enumerate}
    其中第2条等价于$\forall \epsilon$,$\exists x_0 \in R$,使$x_0 > \beta - \epsilon$
    
    则称$\beta $为$E$的上界,记为$\beta = \sup E$
\end{definition}

相应下界记为$\alpha = \inf E$

如果有最大、最小值和上、下确界则它们对应相等。

\begin{thm}
    确界定理:$R$中有上界或下界的集合,必有对应的确界。
\end{thm}

\begin{thm}
    单调有界定理:单调增(或减),有上界(下界)的数列一定收敛。

    单调有界域必收敛。
\end{thm}

\subsection{度量结构 (实数是一个完整的度量空间)}


\begin{definition}
    柯西数列:$\forall \epsilon$,$\exists N > 0 $当$n,m > N$时,有$|x_n - x_m| < \epsilon$
    
    或:$\forall \epsilon$,$\exists N > 0 $当$n > N$时,对所有$p > 0$,有$|x_n - x_{m + p}| < \epsilon$
\end{definition}

\begin{thm}
    柯本收敛准则:$R$中任意柯西数列都是收敛的,反之亦然。
\end{thm}

\begin{thm}
    聚点定理:有界数列必有收敛子列。
\end{thm}

\begin{thm}
    区间套定理:设$[ a_n, b_n ] $是$R$区间里的区间列,如果$[a_{n+1},b_{n+1}]
    \subseteq [ a_n, b_n ]$,且$ \lim_{n \to \infty} (b_n - a_n) = 0$,则
    存在唯一实数$\xi \in [a_n, b_n]$,$ n = 1, 2,3 ,\cdots$
\end{thm}

\begin{definition}
    设$\{x_n\}$为实数列,定义$\{x_n\}$的上极限$\varlimsup\limits_{n \to \infty} x_n$,和下
    极限$\varliminf\limits_{n \to \infty} x_n$为
    \begin{enumerate}
        \item $\varlimsup\limits_{n \to \infty} x_n = \inf\limits_{k \geq 1}( \sup\limits_{n \geq k} x_n) $
        \item $\varliminf\limits_{n \to \infty} x_n = \sup\limits_{k \geq 1}( \inf\limits_{n \geq k} x_n) $
    \end{enumerate}
\end{definition}

\section{函数列及函数项级数的收敛性}
\subsection{一致连续和一致收敛}
\begin{definition}
    设$E$是$R$的子集,$f$是$E$到$R$的函数,$x_0 \in E$,若对$\forall \epsilon > 0$
    ,$\exists \delta>0$,使当$x \in E \cup (x_0-\delta,x_0+\delta)$时,有
    $|f(x) - f(x_0) |< \epsilon$则称$f$在$x_0$点连续;若$f$在$E$上每一点都连续,
    则称$f$在$E$上连续.
\end{definition}

\begin{definition}
    设$E$是$R$的子集,$f$是$E$到$R$的函数,若对任意给定的$\epsilon > 0$
    ,存在$\delta > 0$,当$x_1, x_2 \in E$,只要$|x_1 - x_2 |< \delta$,就有
    $| f(x_1) - f(x_2) | < \epsilon$,则称$f$在$E$上一致连续。
\end{definition}

一致连续是整体概念

例 对$\forall a \in (0,1)$,$f(x) = \frac1x $在$[a,1]$上是一致连续的。

一致连续必定连续,反之不一定成立。

\begin{thm}
    康托一致连续定理:设$f(x)$在$[a,b]$上连续,则$f(x)$在$[a,b]$上一致连续。即闭
    区间上的连续函数一定一致连续。
\end{thm}

\begin{definition}
    设$E$是$R$的子集,$\{f_n\}$是定义在$E$上的函数列,$f$是$E$上的函数,如果对
    每一个$x \in E$,有$\lim_{n \to \infty} f_n(x)=f(x) $,则称$\{f_n\}$在$E$
    上收敛于$f$,并称$f$是$\{f_n\}$在$E$上的极限函数
\end{definition}

\begin{definition}
    设$E$是$R$的子集,$\{f_n\}$是在定义在$E$上的函数列,$f$是$E$上的函数,若对任意
    给定的$\epsilon > 0$,存在自然数$N$,使$n > N$对于一切$x \in E$,恒有
    \[| f_n(x) - f(x) | < \epsilon\]
    则称$\{f_n\}$一致收敛于$f$,并记$f_n \Rightarrow f (n \to \infty)$
\end{definition}


\begin{thm}
    设$f_n [a,b] \to R$是函数列,若$f_n$在$[a,b]$上连续,并且一致收敛于$f(x)$,则
    $f(x)$也连续,且
    \[\lim_{x \to x_0} \lim_{n \to \infty}f_n(x) =
    \lim_{n \to \infty}\lim_{x \to x_0} f_n(x)\]
\end{thm}

\begin{thm}
    设$f_n [a,b] \to R$是函数列,若$f_n$在$[a,b]$上可积,并且一致收敛于$f(x)$,则
    $f(x)$也可积,且
    \[\int_a^b \lim_{n \to \infty}f_n(x) =
    \lim_{n \to \infty}\int_a^b f_n(x) = \int_a^b f(x)\]
\end{thm}

\begin{thm}
    设$f_n [a,b] \to R$是函数列,若
    \begin{enumerate}
        \item $f_n$一致收敛于$f(x)$
        \item $f_n$连续可微(可微导函数连续)
        \item $f_n$的导函数一致收敛于$g(x)$
    \end{enumerate}
    则$f(x)$也可微并且$f'(x) = g(x)$即
    \[ \frac{d}{dx} \lim_{n \to \infty} f_n(x) = \lim_{n \to \infty} 
    \frac{d}{dx} f_n(x)\]
\end{thm}

\section{Lebesgue积分}

黎曼积分是将$[a,b]$区间做分割。

设$I = \lim_{n \to \infty} \sum^{n}_{j=1}  f(\epsilon_j) \triangle x$ 若此值对任意划分都存在且相等
则记$I = \int_a^b f(x) dx$

但对有些函数并不能求出其导,例如
\[
    D(x) =
    \begin{dcases}
        1, & x\text{是有理数} \\
        0, & x\text{是无理数}
    \end{dcases}
\]

而Lebesgue积分的主要思想是分y轴。
\[
    E_i = \{X \in (a,b)| y_{i-1} \le f(x) < y_i \}
\]
\[
    E_i = \{X \in (a,b)| f(x) \le y_{i} \} / \{X \in (a,b)| f(x) \le y_i \}
\]

\subsection{Lebesgue测度}

区间测试有四种情况:$[a,b]$,$(a,b)$,$(a,b]$,$[a,b)$,$mI$为其区间的测度。

\subsubsection{有界集}
\begin{definition}
    开集:设$G \subseteq R, a \in G$,如果$\exists \epsilon > 0$,使
    $(a,a +\epsilon) \subseteq G$,
    则称$a$为$G$的内点,称$G$的内战的全体为$G$的内部,
    记为 $\overset{\circ}G$,或$intG$

    若$G$的每个点都是其内点,则$G$为开集。
\end{definition}

开集的性质
\begin{enumerate}
    \item $R$和$\emptyset$是开集
    \item 任意开集的并,是开集。
    \item 有限个开集的交集是开集
\end{enumerate}

开集的构造:任何一个开集都可以表示为至多可列个互不相交的构成区间的并集。

构成区间指:
\begin{enumerate}
    \item $(\alpha ,\beta) \in G$
    \item $\alpha , \beta \notin G$
    \item $i \neq j$时,$(\alpha_i, \beta_i) \cap (\alpha_j, \beta_j) = 
        \emptyset$
\end{enumerate}
此时有
\[
    G = \bigcup_{n=1}^{\infty}(\alpha_i, \beta_i)
\]

开集的测度:设$G \subseteq R$为有界开集,$G \subseteq (a,b)$,$mG = \sum^{\infty}_{i=1}
(\beta_i- \alpha_i) < (b-a)$是收敛的

\begin{definition}
    设$F \subseteq (a,b)$,是$R$中的有界集,如果$F^c$是开集,则称$F$是闭集。
\end{definition}

$F$是闭集$\Leftrightarrow$ $\forall x_n \in F$,如果$x_n \rightarrow x$,则
$x \in F$

闭集的性质:
\begin{enumerate}
    \item $\emptyset$,$R$是闭集
    \item 任意闭集的交集是闭集
    \item 有限多个闭集的并集是闭集
\end{enumerate}

闭集的测度:设$F \subseteq (a,b)$是闭集,$mF = b-a - mF^c$
\end{document}
