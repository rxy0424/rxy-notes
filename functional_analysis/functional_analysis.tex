\documentclass[UTF8]{ctexbook}

\usepackage{theorem}
\usepackage{amssymb}

\newtheorem{definition}{定义}[chapter]
\newtheorem{thm}{定理}[chapter]

\newcommand{\jhshi}[1]{\overline{\overline{#1}}}

\title{任昕旸同志的泛函分析笔记}
\author{任昕旸}
\date{\today}

\begin{document}
\maketitle
\tableofcontents

\chapter{预备知识}
\section{集合的一般知识}
\subsection{集合及其运算}
特殊集合有全集和空集,集合的关系有交、并补三种。

一个比较重要的公式,De Morgan公式:
\begin{equation}
    (\bigcup_{i \in I} A_i)^C = \bigcap_{i \in I} A_i^C
\end{equation}

\begin{equation}
    (\bigcap_{i \in I} A_i)^C = \bigcup_{i \in I} A_i^C
\end{equation}

还有直积
\begin{equation}
    \prod_{k=1}^n A_k = \{ (x_1,x_2,\cdots,x_n)|x_k \in A_k,k=1,2,3,\cdots,n \}
\end{equation}

\begin{definition}
    设$A$,$B$是两个集合,如果存在由$A$到$B$上的一对一映射$T$,则说$A$与$B$成
    一一对应或称$A$与$B$对等,记为$A \sim B$。
\end{definition}

由定义可知其有三个基本性质:
\begin{enumerate}
    \item 自反性:$A \sim A$
    \item 对称性:若$A \sim B$,则$B \sim B$
    \item 传递性:若$A \sim B, B \sim C$,则$A \sim B$
\end{enumerate}

\subsection{可列集}
\begin{definition}
    若$A$与$B$对等,则说$A$与$B$有相同的势(或基数),记$A$的势为$\overline{\overline A}$,
    $B$的势为$\overline {\overline {B}}$,则$\overline{\overline {A}} = \overline{\overline {B}}$
\end{definition}


\begin{definition}
    对$\alpha \in \Gamma$,称$\{ A_{\alpha} \}_{\alpha \in \Gamma}$为定义在$\Gamma$
    上的集族,若$\Gamma = N$,则称其为集列。
\end{definition}

\begin{definition}
    集族的交
    \begin{equation}
    \bigcap_{\alpha \in \Gamma} A_{\alpha} 
    = \{ x \in X \mid \forall \alpha \in \Gamma ,
        x \in A_{\alpha} \}
    \end{equation}
\end{definition}

\begin{definition}
    集族的并
    \begin{equation}
    \bigcup_{\alpha \in \Gamma} A_{\alpha} 
    = \{ x \in X \mid \exists \alpha \in \Gamma ,
    x \in A_{\alpha} \}
    \end{equation}
\end{definition}

\begin{thm}
    无限集必与它的某个真子集对等.
\end{thm}

\begin{definition}
    凡与自然数集$N$对等的集合称为可列集(或可数集).
\end{definition}

\begin{thm}
    任意无限集必含有可列子集.
\end{thm}

\begin{thm}
    有限集或可列个可列集的并仍是可列集;可列个有限集的并(若为无限集),也是可列集.
\end{thm}

\begin{thm}
    有一组可列集$\{ A_1, A_2, \cdots, A_n \}$,则直积$ \prod_{k=1}^n A_k $也是可列集.
\end{thm}

\begin{thm}
    点集$ ( 0, 1)  = \{ x \in R \mid 0 <  x <  1 \} $是不可列的无限集.
\end{thm}

\begin{thm}
    设$A$为有限集可列集,$B$是任一无限集,则$\jhshi{A \cup B} = \jhshi B$
\end{thm}


\end{document}
